\documentclass[hyperref,UTF8,12pt,a4paper]{ctexart}

\usepackage{amsmath}

\usepackage{geometry}
\geometry{left=1in,right=1in,top=1in,bottom=1in}

\hypersetup{
	colorlinks=true,
	linkcolor=black
}

\usepackage{titling}
\pretitle{\begin{center}\fontsize{30pt}{30pt}\selectfont}
\posttitle{\end{center}}

\usepackage{ulem}

\title{生活中的数学}
\author{罗江楠(swwind)}
\date{2020.10.2}

\begin{document}
\maketitle

\tableofcontents

\newpage

\section{前言}

数学与我们的生活息息相关,每一次科学的进步都离不开数学的发展,每一次数学的发展也离不开一代代数学家们的辛勤贡献。

每天早上醒来之后的第一件事,也许是拿起手机关闭闹钟的声响。手机,这一件对于大多数人来说已经是生活必备品的电子产品,中间融入了多少科技,凝结了多少代人的思维结晶,最终才形成了现在的模样。

本文将从普通智能手机作为切入点,宽泛地介绍目前生活中被广泛应用的科学技术中的数学原理。

\section{加密通讯}

现代所有的加密通讯都离不开密码学的支持,而密码学的内容大多依赖于数论。数学家高斯曾经高度评价说:“数学是科学的皇后,数论是数学的皇后。”数论研究的是整数的性质,被誉为“最纯”的数学领域。

\subsection{RSA 算法}

RSA 是一种非对称加密算法,早已在商业领域被广泛利用。下面简要介绍其基本过程。

\subsubsection{过程}

假设 Alice 想要跟 Bob 发送私人消息,那么 Alice 要先找到两个质数 $p$ 和 $q$,并计算 $N=pq$ 以及 $r=\varphi(N)=(p-1)(q-1)$。

接着,找一个小于 $r$ 并且与 $r$ 互质的数 $e$,并且求出 $e$ 在 $r$ 模域下的逆元 $d$($ed \equiv 1 \pmod r$)。

这时,$(N, e)$ 即为公钥,$(N, d)$ 即为私钥,Alice 将他的公钥 $(N, e)$ 发送给 Bob,同时保密好自己的私钥 $(N, d)$。

如果 Bob 想要给 Alice 发送消息,则先通过约定手段将数据转化成一个小于 $N$ 的非负整数 $n$,接着通过下式求得加密后的数据 $c$。

$$
c \equiv n^e \pmod N
$$

接着,Bob 可以将 $c$ 发送给 Alice,Alice 收到消息之后可以通过下式求得加密前的数据 $n$。

$$
n \equiv c^d \pmod N
$$

\subsubsection{原理}

\textbf{欧拉定理:}

设 $m > 1, (a, m) = 1$,则

$$
a^{\varphi(m)} \equiv 1 \pmod m
$$

于是我们可以知道

$$
(c^e)^d \equiv c^{ed} \equiv c^{\varphi(N)} \times c \equiv 1 \times c \equiv c \pmod N
$$

对于中间攻击者来说,如果想要窃听 Bob 发送给 Alice 的消息,那么就必须获取 Alice 手中的私钥,也就是需要将公钥中的 $N$ 分解成两个质数 $p$ 和 $q$ 的乘积。

由于目前尚未找到高效率的大数分解质因数的办法,因此如果 Alice 找的 $p$ 和 $q$ 足够大,那么足以保证通讯的安全性。

事实上人们已经发现了基于量子计算机的分解质因数的 Shor 算法,其理论复杂度为 $O((\log N) ^ 3)$,如果量子计算机真的得以实现,那么 RSA 算法就危险了。

\subsection{D-H 密钥交换协议}

如果说 RSA 算法的精髓在于分解大数质因数,那么 D-H 密钥交换协议的精髓就在于模域下求对数。

\subsubsection{过程}

假设 Alice 和 Bob 需要通过一个不安全的信道商定一份对称性加密算法的密钥,那么 Alice 需要先找一个质数 $p$ 和他的原根 $g$,并且发送给 Bob。

接着,Alice 选择一个随机正整数 $a$,计算 $A=g^a \pmod p$ 并将 $A$ 发送给 Bob。Bob 也选择一个随机正整数 $b$,计算 $B=g^b \pmod p$ 并将 $B$ 发送给 Alice。

这时 Alice 计算 $C = B^a \pmod p$,Bob 也计算 $C = A^b \pmod p$,至此两人获得相同的数字 $C$。

\subsubsection{原理}

显然在模 $p$ 意义下 $g^{ab}$ 与 $g^{ba}$ 相等。

对于中间攻击者来说,想要仅仅通过 $A$ 和 $B$ 来求得 $a$ 和 $b$ 是很困难的,因此只要 $p,a,b$ 足够大,那么就能保证通讯的安全性。

\section{无线通讯}

网络通讯是所有通讯设备最基础的功能和设计目的。基础的无线通信基于对于电磁波的调制和解调,其中对于波的解调就涉及傅里叶变换。

在大气层中的电磁波是十分混乱的。所有电子设备发射的不同波段的电磁波都混杂在一起,形成了一个杂乱无章的波形。然而电子设备却可以从这个杂乱的波形中提取出自己需要的波段,从而互相不干扰。

下面简单抽象地介绍一下其中涉及的傅里叶变换的原理。

\subsection{傅里叶级数}

假设我们有一个周期为 $2\pi$ 的函数 $f(x)$,且 $f(x)$ 可以在 $[-\pi,\pi]$ 上可积,那么 $f(x)$ 可以展开成三角级数的形式

$$
f(x)=\frac{a_0}{2}+\sum_{k=1}^{\infty}(a_k\cos{kx}+b_k\sin{kx})
$$

这个三角级数也被称为傅里叶级数。

这个级数的意义十分显然:一个波形必定是由许多不同波段的正弦波组成的。

那么如何求上式中的 $a_0, a_k$ 和 $b_k$ 呢?也就是说,如何将一个波形中 $\sin{kx}$ 的成分提取出来呢?

首先我们知道,三角函数系

$$
1, \cos x, \sin x, \cos 2x, \sin 2x, ..., \cos nx, \sin nx, ...
$$

在区间 $[-\pi, \pi]$ 上正交,即函数系中任意两个不相同的函数的乘积在区间 $[-\pi, \pi]$ 上的积分等于 $0$,例如

$$
\begin{aligned}
\int_{-\pi}^{\pi}\sin x {\rm d} x &= 0 \\
\int_{-\pi}^{\pi}\cos 5x \sin 2x {\rm d} x &= 0 \\
\int_{-\pi}^{\pi}\cos 3x \cos 7x {\rm d} x &= 0 \\
\cdots &
\end{aligned}
$$

但是函数系中两个相同的函数的乘积在区间 $[-\pi, \pi]$ 上的积分却不等于 $0$,即

$$
\begin{aligned}
\int_{-\pi}^{\pi}1^2 {\rm d} x &= 2\pi \\
\int_{-\pi}^{\pi}\sin^2 nx {\rm d} x &= \pi, (n = 1, 2, 3, ...) \\
\int_{-\pi}^{\pi}\cos^2 nx {\rm d} x &= \pi, (n = 1, 2, 3, ...) \\
\end{aligned}
$$

那么很显然,我们可以直接将 $f(x)$ 与 $\sin kx$ 相乘然后求积分,得出的结果即为 $\pi a_k$。

$$
a_k = \frac{1}{\pi}\int_{-\pi}^{\pi}f(x)\sin kx {\rm d}x
$$

想要求得 $\cos kx$ 的系数 $b_k$ 也是同理,这里不再赘述。

求出了 $\sin kx$ 以及 $\cos kx$ 之后我们就可以通过合一变形公式求出 $\sin kx$ 波形成分的偏移 $\varphi$ 和幅度 $A$。

\section{总结}

我们的生活离不开数学,数学与我们的生活息息相关,因此学好数学是我们应当一生悬命追求的目标。

\newpage

\bibliographystyle{plain}

\begin{thebibliography}{99}
\bibitem{a} RSA (cryptosystem), Wikipedia, \\
\texttt{https://en.wikipedia.org/wiki/RSA\_(cryptosystem)}
\bibitem{b} Diffie–Hellman key exchange, Wikipedia, \\
\texttt{https://en.wikipedia.org/wiki/Diffie–Hellman\_key\_exchange}
\bibitem{c} 同济大学数学系(2014). 高等数学 第七版. 高等教育出版社.
\end {thebibliography}

\end{document}

\documentclass[hyperref,UTF8,12pt,a4paper]{ctexart}
\usepackage{amsmath}
\usepackage{amsfonts}

\usepackage{geometry}
\geometry{left=1in,right=1in,top=1in,bottom=1in}

\hypersetup{
	colorlinks=true,
	linkcolor=black
}

\usepackage{titling}
\pretitle{\begin{center}\fontsize{30pt}{30pt}\selectfont}
\posttitle{\end{center}}

\usepackage{fancyhdr}
\pagestyle{fancy}
\fancyhf{}
\fancyhfoffset[L]{0cm} % left extra length
\fancyhfoffset[R]{0cm} % right extra length
\chead{哈尔滨工业大学《数学与科技进步》学习报告}
\rfoot{}

\usepackage{ulem}

\title{简单群论与五次以上方程的根}
\author{罗江楠}
\date{2021.5.5}

\begin{document}
\maketitle

% \newpage

% \tableofcontents

\newpage

\section{摘要}

% 许多数学家费劲了心思想寻找五次方程的根,但是最终都无果告终。一切的努力似乎都在说明一件事:五次以上的方程也许根本没有求根公式。但是如何证明呢?

求一元五次方程的根式解曾困扰数学家三百余年,其中有不少数学家尝试证明其无解,但最终都以失败告终。最终,这个问题在 19 世纪由阿贝尔和伽罗瓦运用崭新的数学概念——群论,才完整地证明了一般一元五次方程没有根式解。

本文将从简单群论开始简要阐述伽罗瓦理论以及如何用来证明五次及以上的方程没有求根公式。

\newpage

\section{五次以下方程的求根公式}

对于五次以下的方程数学家们老早就帮我们找好了求根公式。

\subsection{一次方程}

对于一元一次方程 $ax+b=0$ 易知有解 $x=-\frac ba$。

\subsection{二次方程}

对于一元二次方程 $ax^2+bx+c=0$(其中 $a,b,c\in\mathbb R,a\neq 0$),根据代数基本定理可知,该方程有两个复数根,设其为 $x_1,x_2$,于是我们可以将方程改写为 $a(x-x_1)(x-x_2)=0$,即 $ax^2-a(x_1+x_2)x+ax_1x_2=0$,与原式联立得

$$
\begin{cases}
x_1+x_2=-\frac ba \\
x_1x_2=\frac ca
\end{cases}
$$

因此有

$$
\begin{aligned}
(x_1-x_2)^2&=(x_1+x_2)^2-4x_1x_2\\
&=(-\frac ba)^2-4\frac ca \\
&=\frac{b^2-4ac}{a^2}
\end{aligned}
$$

可得

$$
x_1-x_2=\pm\frac{\sqrt{b^2-4ac}}{a}
$$

又已知

$$
x_1+x_2=-\frac ba
$$

联立得

$$
x_1,x_2=\frac{-b\pm\sqrt{b^2-4ac}}{2a}
$$

\subsection{三次方程}

对于一元三次方程 $ax^3+bx^2+cx+d=0$,为简化问题,我们将每一项的系数都除以 $a$,得到 $x^3+Ax^2+Bx+C=0$。

将 $x=y-\frac A3$ 带入上式得

$$
\begin{aligned}
(y-\frac A3)^3+A(y-\frac A3)^2+B(y-\frac A3)+C&=0 \\
(y^3-Ay^2+\frac {A^2}3y-\frac {A^3}{27})+(Ay^2-\frac {2A^2}3y+\frac{A^3}9)+(By-\frac {AB}3)+C&=0\\
y^3+(B-\frac{A^2}3)y+(\frac{2A^3}{27}-\frac{AB}3+C)&=0
\end{aligned}
$$

从而我们去掉了二次项。令 $p=B-\frac{A^2}3,q=\frac{2A^3}{27}-\frac{AB}3+C$,可简写为

$$
y^3+py+q=0
$$

设原三次方程的其中一个解为 $x_1$,令 $y_1=x_1+\frac A3=m+n$(其中 $m,n$ 为任取),则有

$$
\begin{aligned}
(m+n)^3+p(m+n)+q&=0\\
m^3+n^3+3mn(m+n)+p(m+n)+q&=0\\
(m+n)(p+3mn)+(q+m^3+n^3)&=0
\end{aligned}
$$

假设我们取的 $m,n$ 满足

$$
(m+n)(p+3mn)=q+m^3+n^3=0
$$

可得

$$
\begin{cases}
m^3+n^3=-q \\
mn=-\frac p3
\end{cases}
$$

由于

$$
\begin{aligned}
(m^3-n^3)^2&=(m^3+n^3)^2-4m^3n^3 \\
&=q^2+\frac{4p^3}{27}
\end{aligned}
$$

于是有

$$
m^3-n^3=\pm\sqrt{q^2+\frac{4p^3}{27}}
$$

联立解得

$$
m^3,n^3=-\frac q2\pm\sqrt{\frac{q^2}4+\frac{p^3}{27}}
$$

即

$$
m,n=\sqrt[3]{-\frac q2\pm\sqrt{\frac{q^2}4+\frac{p^3}{27}}}
$$

于是可以得到

$$
\begin{aligned}
x_1&=m+n-\frac A3 \\
&=\sqrt[3]{-\frac q2+\sqrt{\frac{q^2}4+\frac{p^3}{27}}}+\sqrt[3]{-\frac q2-\sqrt{\frac{q^2}4+\frac{p^3}{27}}}-\frac A3
\end{aligned}
$$

对于三次方程 $x^3=1$,可以化为

$$
(x-1)(x^2+x+1)=0
$$

因此可以求得其三个根分别为

$$
\begin{cases}
x_1=1\\
x_2=\frac{-1+\sqrt{3}i}2\\
x_3=\frac{-1-\sqrt{3}i}2
\end{cases}
$$

令 $\omega=x_2=\frac{-1+\sqrt{3}i}2$,则原三次方程的另外两个根即为

$$
\begin{cases}
x_2=\omega m+\omega^2n-\frac A3 \\
x_3=\omega^2m+\omega n-\frac A3
\end{cases}
$$

\subsection{四次方程}

同上,对于四次方程 $x^4+ax^3+bx^3+cx^2+dx+e=0$,将 $x=y-\frac a4$ 带入可以消去三次项。将带入后所得的新方程设为 $y^4+Ay^2+By+C=0$,将其写成两个二次方程的乘积

$$
(y^2+ky+l)(y^2+my+n)=0
$$

展开后得

$$
y^4+(k+m)y^3+(l+km+n)y^2+(lm+kn)y+ln=0
$$

对比系数得

$$
\begin{cases}
k+m=0 \\
l+km+n=A \\
lm+kn=B \\
ln=C \\
\end{cases}
$$

解方程组得

$$
m=-k \\
\Rightarrow
\begin{cases}
l+n=A+k^2 \\
n-l=\frac Bk \\
\end{cases} \\
\Rightarrow
\begin{cases}
2n=A+k^2+\frac Bk \\
2l=A+k^2-\frac Bk \\
\end{cases} \\
\Rightarrow
(A+k^2+\frac Bk)(A+k^2-\frac Bk)=4C
$$

化简即得

$$
k^6+2Ak^4+(A^2-4C)k^2-B^2=0
$$

这是一个关于 $k^2$ 的一元三次多项式,因此 $k^2$ 可解,即 $k$ 可解,从而 $l,m,n$ 均可解。

易知四次方程的其中两个根 $y_1,y_2$ 为二次方程 $y^2+ky+l=0$ 的两个根,$k, l$ 均可解,于是 $y_1,y_2$ 也均可解。

同理,另外两个根 $y_3,y_4$ 满足 $y^2+my+n=0$,均可解。

至此原四次方程的四个根都可解。

最后的结果过于复杂此处略去。

\section{群论}

\subsection{定义}

一个群由一个集合 $G$ 和一个二元运算(一般为 “$\cdot$”)构成,并且要满足以下四个群公理:

\begin{itemize}
\item 封闭性:对于 $\forall a,b \in G$,满足 $a \cdot b \in G$;
\item 结合律:对于 $\forall a,b,c \in G$,满足 $a \cdot (b \cdot c)= (a \cdot b) \cdot c$;
\item 单位元:存在唯一 $e \in G$ 使得对于 $\forall a \in G$ 满足 $a \cdot e = a$;
\item 逆元:对于 $\forall a \in G$,存在 $a^{-1} \in G$ 满足 $a \cdot a^{-1}=e$。
\end{itemize}

比如,所有整数($\mathbb Z$)和加法操作满足群公理,因此可以称为一个群,我们将其称为整数加法群,记为 $\langle\mathbb Z, +\rangle$。

同样道理的还有 $\langle\mathbb Q, +\rangle$、$\langle\mathbb R, +\rangle$ 以及 $\langle\mathbb C, +\rangle$。

值得注意的是 $\langle\mathbb R, \times\rangle$ 并不是群,因为 $0$ 没有逆元。$\langle\mathbb R - \{0\}, \times\rangle$ 则是一个群。

群的阶即为集合的元素个数,记为 $|G|$。

\subsection{阿贝尔群}

对于一个群 $G$,如果对于 $\forall a,b \in G$ 都满足 $a \cdot b = b \cdot a$(交换律),则称该群为\textbf{阿贝尔群}。

\subsection{置换,轮换与对换}

对于一个仅由三个元素构成的排列 $(1, 2, 3)$,我们可以知道其存在 $6$ 种不同的置换并且可以组成一个群,我们将这个群记为 $S_3$。

\begin{table}[ht]
\centering
\begin{tabular}{c c c}
\hline
置换 & 记号 & 结果 \\
\hline
什么也不干        & $e$   & $(1, 2, 3)$ \\
所有元素左移一位   & $a_1$ & $(2, 3, 1)$ \\
所有元素左移两位   & $a_2$ & $(3, 1, 2)$ \\
对换第一个和第二个 & $b_1$ & $(2, 1, 3)$ \\
对换第一个和第三个 & $b_2$ & $(3, 2, 1)$ \\
对换第二个和第三个 & $b_3$ & $(1, 3, 2)$ \\
\hline
\end{tabular}
\end{table}

其中每两种置换都可以复合,即先执行第一个再执行第二个。

如果置换 $h$ 等价于先执行 $f$ 后执行 $g$,则我们记为 $h=g \cdot f$。

注意这里顺序要反着写,因为如果我们将置换看成函数的话就相当于 $h(x)=g(f(x))$,因此先执行的置换要写在右边。

我们将仅交换两个元素的置换称为\textbf{对换},易知对于 $S_3$ 中的所有置换,都可以通过不同的对换复合得到。

$$
\begin{cases}
e&=b_1 \cdot b_1 = b_2 \cdot b_2 = b_3 \cdot b_3 \\
a_1&=b_3 \cdot b_1 = b_1 \cdot b_2 \\
a_2&=b_2 \cdot b_1 = b_1 \cdot b_3 \\
b_1&=b_1 \\
b_2&=b_2 \\
b_3&=b_3 \\
\end{cases}
$$

经过简单的观察发现,经过奇数次对换后的置换永远只能通过奇数次对换得到,偶数同理。

于是我们将通过奇数次对换后得到的置换称为\textbf{奇置换},通过偶数次对换后得到的置换称为\textbf{偶置换}。

显然对于 $n$ 次的置换群 $S_n$,其总共拥有的 $n!$ 种不同的置换中总有一半是奇置换,一半是偶置换。

其中所有的偶置换可以形成一个新的群,我们将其称为\textbf{交错群},记为 $A_n$。

\begin{quote}
曾经有一道经典的数字华容道谜题(仅对换了 $14$ 和 $15$),我们来证明其为何无解。

\begin{verbatim}
1  2  3  4
5  6  7  8
9  10 11 12
13 15 14 __
\end{verbatim}

我们将右下角的空缺看做 $16$,则每次操作相当于将 $16$ 与周围的数字进行一次对换。

将原棋盘黑白染色,由于初始 $16$ 的位置和结束状态 $16$ 的位置相同,因此易知结束的局面必然是将 $16$ 对换了偶数次的结果(偶置换)。

由于初始局面仅对换了 $14$ 和 $15$,是个奇置换,因此无论如何也无法从结束状态经过偶数次置换到达奇置换的初始状态。

因此此题无解。
\end{quote}

\begin{quote}
将一个普通三阶魔方强行扭转一个角块之后形成的局面无法复原,但是再扭转一次角块(无论是不是原来扭转的角块)之后便可以复原的原理也是类似。
\end{quote}

\subsection{循环群}

对于一个群 $G$,如果对于任意元素 $a \in G(a \ne e)$,使得 $G$ 中的所有元素都可以通过 $a$ 反复自我复合得到,则称群 $G$ 为一个\textbf{循环群}。

下面证明一个素数阶的群 $G$ 必然是一个循环群。

\begin{quote}
\textbf{拉格朗日定理}:对于群 $G$ 的任意一个子群 $H$($H \subseteq G$), 群 $H$ 的阶必定为群 $G$ 的阶的一个约数。

由拉格朗日定理可知,对于一个\textbf{素数阶}的群 $G$,不存在除了其本身和\textbf{幺群}($\{e\}$)之外的子群。

我们从群 $G$ 中任意找出一个元素 $a$,用其构造一个循环群 $H = \{e, a, a^2, a^3, ..., a^{-1}\}$。

由群公理可知群 $H$ 必定为群 $G$ 的一个子群。

\begin{itemize}
\item 如果 $a = e$,则子群 $H=\{e\}$,是幺群。
\item 如果 $a \ne e$,则子群 $H$ 的阶必定大于 $1$,不是幺群,只能是群 $G$ 本身,即 $H=G$。
\end{itemize}

由于素数 $> 1$,因此必定能找到元素 $a \in G (a \ne e)$ 能够构造出与群 $G$ 同构的循环群 $H$,因此群 $G$ 必定是循环群。

至此原命题得证。
\end{quote}

下面证明每个循环群都是阿贝尔群。

\begin{quote}
对于一个循环群 $H=\{e, a, a^2, ..., a^n\}$,其中的每个元素都可以表示成 $a^x(x\in{0,1,2,...,n})$ 的形式。

由群公理中的结合律可知 $a^x \cdot a^y = a^{x+y} = a^y \cdot a^x$,因此满足交换律,是阿贝尔群。
\end{quote}

\subsection{正规子群}

对于一个群 $G$,设其存在子群 $H$,使得对于 $\forall a \in G$,都满足 $a \cdot H = H \cdot a$,则我们称子群 $H$ 为群 $G$ 的一个\textbf{正规子群},记为 $G \Delta H$。

元素与集合的运算定义为 $a \cdot \{b_1, b_2, ..., b_n\} = \{a \cdot b_1, a \cdot b_2, ..., a \cdot b_n\}$,左右交换同理。

易知群 $G$ 本身与\textbf{幺群}都是群 $G$ 的\textbf{正规子群},定义这两个群是群 $G$ 的\textbf{平凡正规子群}。

所有\textbf{阿贝尔群}的所有子群都是\textbf{正规子群}。

\subsection{商群}

对于群 $G$ 的一个\textbf{正规子群} $H$,定义\textbf{商群}为 $G$ 中所有元素分别与 $H$ 运算后形成的群,记为 $G/H$,且商群的阶 $|G/H|=\frac{|G|}{|H|}$。

易知任意群对幺群的商群都是其本身,即 $G/\{e\}=G$。

\subsection{单群}

对于群 $G$,如果不存在除了\textbf{平凡正规子群}之外的其他\textbf{正规子群},则称群 $G$ 为\textbf{单群}。

以下几种群都是单群:

\begin{itemize}
\item 素数阶循环群
\item $n \ge 5$ 的交错群
\item $...$
\end{itemize}

此外还有许多种类的单群,具体可以参考有限单群分类定理(这个定理的证明有上万页!)。

\subsection{可解群}

对于一个群 $G$,我们找到一个\textbf{正规子群列}形如

$$
G \Delta G_1 \Delta G_2 \Delta G_3 \Delta ... \Delta G_n \Delta \{e\}
$$

其中每一个群都是前一个群的正规子群。

如果这个正规子群列满足每两个相邻的群的商群都是阿贝尔群,则称群 $G$ 是\textbf{可解群}。

易知所有的阿贝尔群都是可解群。

\section{伽罗瓦理论}

\subsection{伽罗瓦说了什么}

一个 $n$ 次方程是否可解,本质上即为方程的 $n$ 个根能否被区分开来。如果一个 $n$ 次置换群是可解群,那么说明一个 $n$ 次方程可解。反之则不可解。

\subsection{具体一点}

对于二次置换群 $S_2$,我们可以发现 $S_2$ 与 $C_2$ 同构,是阿贝尔群,因此可知 $S_2$ 是可解群,二次方程可解。

对于三次置换群 $S_3$,我们可以找到 $S_3 \Delta A_3 \Delta \{e\}$。

\begin{itemize}
\item 对于 $S_3/A_3$,它的阶是 $2$,是一个素数,因此这个商群是一个循环群,是阿贝尔群。
\item 对于 $A_3/\{e\}$,我们发现 $A_3$ 与 $C_3$ 同构,而且 $C_3$ 是循环群,是阿贝尔群,因此这个商群也是阿贝尔群。
\item 于是 $S_3$ 是可解群,因此三次方程也可解。
\end{itemize}

对于四次置换群 $S_4$,我们可以找到 $S_4 \Delta A_4 \Delta V \Delta \{e\}$,其中 $V$ 是克莱因四元群($\{(1,2,3,4),(1,2,4,3),(2,1,3,4),(2,1,4,3)\}$)。

\begin{itemize}
\item 对于 $S_4/A_4$,阶为 $2$,是素数阶循环群,是阿贝尔群。
\item 对于 $A_4/V$,阶为 $3$,是素数阶循环群,是阿贝尔群。
\item 对于 $V/\{e\}$,根据定义知 $V$ 是阿贝尔群。
\item 于是 $S_4$ 是可解群,因此四次方程也可解。
\end{itemize}

对于五次置换群 $S_5$,我们可以找到 $S_5 \Delta A_5 \Delta \{e\}$。

\begin{itemize}
\item 对于 $S_5/A_5$,阶为 $2$,是素数阶循环群,是阿贝尔群。
\item 对于 $A_5/\{e\}$,阶为 $60$ 不是素数,商群不是阿贝尔群。
\item 于是 $S_5$ 为不可解群,因此五次方程不可解。
\end{itemize}

对于五次以上的置换群 $S_n(n > 5)$,$S_5$ 必为其一个正规子群,$S_5$ 为不可解群,因此 $S_n$ 也必然为不可解群,因此五次以上的方程都没有求根公式。

至此,五次以上方程没有根式解的命题证明完毕。

\newpage

\bibliographystyle{plain}

\begin{thebibliography}{99}
\bibitem{a} 有限单群分类, Wikipedia, \\
\texttt{https://zh.wikipedia.org/wiki/\%E6\%9C\%89\%E9\%99\%90\%E5\%96\%AE\%E7\%BE\%A4\%E5\%88\%86\%E9\%A1\%9E}
\bibitem{b} 伽罗瓦理论, Wikipedia, \\
\texttt{https://zh.wikipedia.org/wiki/\%E4\%BC\%BD\%E7\%BE\%85\%E7\%93\%A6\%E7\%90\%86\%E8\%AB\%96}
\bibitem{c} 阿贝尔群, Wikipedia, \\
\texttt{https://zh.wikipedia.org/wiki/\%E9\%98\%BF\%E8\%B4\%9D\%E5\%B0\%94\%E7\%BE\%A4}
\bibitem{d} 群论, Wikipedia, \\
\texttt{https://zh.wikipedia.org/wiki/\%E7\%BE\%A4\%E8\%AE\%BA}
\bibitem{e} 单群, Wikipedia, \\
\texttt{https://zh.wikipedia.org/wiki/\%E5\%8D\%95\%E7\%BE\%A4}
\bibitem{f} 五次方程为什么没有求根公式?, 1-6, 妈咪说MommyTalk, YouTube \\
\texttt{https://youtu.be/CdBbPkXxc3E?list=PLYtoePJQbGmiq3lqkJcxSKW0eW5objmjQ}

\end {thebibliography}

\end{document}

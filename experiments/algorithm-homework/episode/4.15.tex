\section*{4.15}

贪心的策略是:按照 $b_i$ 的大小从大到小排序,接着一次将每个零件放到机器 A 上进行预处理,并在完成之后立刻放到机器 B 上处理。

下面证明这个贪心策略的正确性。

我们令 $f(i)$ 为零件 $i$ 完成所需要的时间,即 $f(i) = \sum_{j=1}^{i}a_j + b_i$,显然最终答案为 $\max_{i=1,2,...,n}f(i)$。

假设 $\exists i \in N, 1 \le i < n$ 使得 $b_i < b_{i+1}$,那么我们可以通过交换零件 $i$ 和零件 $i+1$ 使得答案变得更优。

\begin{proof}
  可以发现,将零件 $i$ 和零件 $i+1$ 交换前后,除了这两个零件外其他零件的完成时间都不会发生变化。因此我们可以只关注零件 $i$ 和零件 $i+1$ 的完成时间的变化。

  由计算可知交换前

  $$
  \begin{aligned}
  f(i)   &= \sum_{j=1}^{i-1} a_j + a_{i} + b_{i} \\
  f(i+1) &= \sum_{j=1}^{i-1} a_j + a_{i} + a_{i+1} + b_{i+1}
  \end{aligned}
  $$

  交换后

  $$
  \begin{aligned}
  f'(i)   &= \sum_{j=1}^{i-1} a_j + a_{i+1} + a_{i} + b_{i} \\
  f'(i+1) &= \sum_{j=1}^{i-1} a_j + a_{i+1} + b_{i+1}
  \end{aligned}
  $$

  并且由于 $b_{i} < b_{i + 1}$,因此我们有

  $$
  \begin{aligned}
  \max\{f(i), f(i+1)\} &= f(i+1) \\
  f'(i) &< f(i+1) \\
  f'(i+1) &< f(i+1)
  \end{aligned}
  $$

  故

  $$
  \max\{f'(i), f'(i+1)\} < f(i+1) = \max\{f(i), f(i+1)\}
  $$

  成立,即交换之后的答案更优。
\end{proof}

由于每次交换逆序对都可以使答案更优,反之则更劣,因此我们可以将每个相邻的逆序对都进行交换,直到整个序列变得有序,而这必然是我们要找的最优解。

整体算法依赖排序,因此最坏时间复杂度为 $O(n \log n)$。

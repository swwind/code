\section*{4.2}

使用贪心算法,设计思想是每次从所有的物品中选择一个 $l_i$ 最小的,并将它加入到答案中,如果某时刻答案的总长度无法再容下剩余的任意一个物品,则算法结束,当前答案即为最终答案。

下面证明贪心算法对于任意的 $L \in N$ 均为最优解。

\begin{proof}
  不失一般性,我们可以设 $l_1 \le l_2 \le l_3 \le ... \le l_n$。

  假设当 $L = k (k \in N)$ 时,满足 $S \in 2^{\{1,2,...,n\}}, \sum_{s \in S} l_{s} \le k$ 的所有 $S$ 中元素个数最多的集合为 $A$,这显然是我们要找的最优解。

  如果 $A = \{1,2,3,...,|A|\}$,则说明我们利用贪心构造出来的解就是最优解。

  如果 $A \ne \{1,2,3,...,|A|\}$,则我们令 $p = |A|$,$B = \{1,2,3,...,|A|\}$,显然我们可以发现 $\forall i, 1 \le i \le n$ 都有 $B_i \le A_i$ 成立。因此我们可以得到集合 $B$ 的答案

  $$
  \sum_{i=1}^{p}l_{B_i} \le \sum_{i=1}^{p} l_{A_i} \le k = L
  $$

  因此集合 $B$ 肯定也为 $L=k$ 时的最优解。

  因此我们利用贪心的算法必定能够构造出当前问题的最优解。
\end{proof}

时间复杂度为 $O(n \log n)$。


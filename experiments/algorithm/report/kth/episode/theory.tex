\section{算法原理}

\subsection{快速选择算法}

快速选择算法类似于快速排序算法,同样是先从整个数组中随机选择一个元素作为基准值,然后将所有小于这个基准值的元素都移动到数组的 前面,所有大于这个基准值的元素都移动到数组的后面。之后统计这个基准值在数组中出现的范围 $[l, r)$,如果要查询的元素的序号恰好在这个区间内,则说明要找的元素就是这个基准值。否则分别在左右两个子数组中递归地查找。

复杂度为均摊 $O(n)$,最坏情况为 $O(n^2)$。

\subsection{分组选择算法}

这个算法的思想是,先将整个数组按照五个一组排序,接着在这些分组中找出各自的中位数,接着将这些中位数中的中位数寻找出来,作为一个基准值。接下来将原数组按照与基准值的大小分为大于组和小于组,并按照这两个数组的大小计算出基准值在原数组中出现的范围。然后与上一种算法类似,如果要找的元素序号恰好在基准值的范围内,则说明找到了,否则分别在大于小于两个子数组中递归地查找。

复杂度 $T(n) = \frac{n}{5} T(5) + T(\frac{n}{5}) + O(n) + T(\frac{n}{2}) = O(n)$。

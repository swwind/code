\section{算法程序}

首先我们有定义排序算法的 trait。

\begin{minted}{rust}
pub trait Sort<T> {
  fn sort(v: &mut [T]) -> ();
}
\end{minted}

\subsection{快速排序算法}

\begin{minted}{rust}
pub fn pivot<T>(v: &mut [T]) -> (usize, usize)
where
  T: Ord + Copy,
{
  let n = v.len();
  let x = rand::random::<usize>() % n;
  let baseline = v[x];

  let (mut l, mut r) = (0, n);

  for i in 0..n {
    if v[i] < baseline {
      v.swap(i, l);
      l += 1;
    }
  }

  for i in (0..n).rev() {
    if v[i] > baseline {
      v.swap(i, r - 1);
      r -= 1;
    }
  }

  (l, r)
}

pub struct QuickSort;

impl<T> Sort<T> for QuickSort
where
  T: Ord + Copy,
{
  fn sort(v: &mut [T]) -> () {
    if v.len() < 2 {
      return;
    }

    let (l, r) = pivot(v);

    Self::sort(&mut v[..l]);
    Self::sort(&mut v[r..]);
  }
}
\end{minted}

\subsection{归并排序算法}

\begin{minted}{rust}
pub struct MergeSort;

impl<T> Sort<T> for MergeSort
where
  T: Ord + Copy,
{
  fn sort(v: &mut [T]) -> () {
    let n = v.len();
    if n < 2 {
      return;
    }

    let (left, right) = v.split_at_mut(n / 2);

    // 递归排序
    Self::sort(left);
    Self::sort(right);

    let mut vec = Vec::with_capacity(n);

    let (mut i, mut j) = (0, 0);

    // 合并左右两半的数组,直到其中一方用尽
    while i < left.len() && j < right.len() {
      if left[i] < right[j] {
        vec.push(left[i]);
        i += 1;
      } else {
        vec.push(right[j]);
        j += 1;
      }
    }

    // 添加左边数组中的剩余部分
    vec.extend_from_slice(&left[i..]);
    // 添加右边数组中的剩余部分
    vec.extend_from_slice(&right[j..]);

    // 复制回原来的数组中
    v.copy_from_slice(&vec);
  }
}  
\end{minted}

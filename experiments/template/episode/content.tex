\section{如何关注}

只要打开 bilibili 就可以直接关注塔菲了喵。

或者你可以点击链接 \url{https://space.bilibili.com/1265680561} 喵。

\section{一些源代码}

\subsection{有序列表}

\begin{enumerate}
  \item 这是第一条消息
  \item 这是第二条消息
\end{enumerate}

\subsection{无序列表}

\begin{itemize}
  \item 这是第一条消息
  \item 这是第二条消息
\end{itemize}

\subsection{数学公式}

$$
E = mc^2
$$

\begin{equation}
  e^{i\pi} + 1 = 0
\end{equation}

\subsection{代码}

\begin{minted}{cpp}
#include <bits/stdc++.h>
int main() {
  int *a = new int[5];
  a[114514] = 1919810;
}
\end{minted}

\subsection{表格}

\begin{tabular}{ cccc } 
  \hline
  \hline
  \textbf{数据重合率} & \textbf{快速排序(ns)} & \textbf{归并排序(ns)} & \textbf{标准库函数(ns)} \\
  \hline
  0\% & \text{113,787,967} & \text{132,552,729} & \text{61,206,861} \\
  10\% & \text{104,529,922} & \text{128,420,548} & \text{61,190,284} \\
  20\% & \text{94,748,819} & \text{122,322,196} & \text{61,071,256} \\
  30\% & \text{84,586,610} & \text{116,958,921} & \text{60,506,152} \\
  40\% & \text{74,765,245} & \text{110,547,765} & \text{58,479,907} \\
  50\% & \text{64,486,606} & \text{103,639,161} & \text{55,395,024} \\
  60\% & \text{54,051,475} & \text{96,579,543} & \text{52,050,725} \\
  70\% & \text{43,410,469} & \text{89,873,605} & \text{47,790,281} \\
  80\% & \text{32,603,312} & \text{81,816,375} & \text{43,602,837} \\
  90\% & \text{22,192,894} & \text{73,799,616} & \text{37,933,618} \\
  100\% & \text{12,468,227} & \text{52,878,853} & \text{12,069,074} \\
  \hline
\end{tabular}

\subsection{画图}

\pgfplotstableread{data.txt}{\table}

\begin{tikzpicture}
  \begin{axis}[
    xmin = 0, xmax = 100,
    ymin = 0, ymax = 150,
    width = 0.8\textwidth,
    height = 0.5\textwidth,
    xlabel = {\text{数据重合率 (\%)}},
    ylabel = {\text{排序时间 (ms)}},
  ]
  \addplot[blue, mark = *] table [x = {x}, y = {y1}] {\table};
  \addplot[red, mark = square] table [x = {x}, y = {y2}] {\table};
  \addplot[green, mark = x] table [x = {x}, y = {y3}] {\table};
  \legend{
    快速排序,
    归并排序,
    标准库排序
  }
  \end{axis}
\end{tikzpicture}

\subsection{伪代码}

\begin{algorithm}
  \caption{快速排序算法}
  \begin{algorithmic}[1]
  \Procedure{QuickSort}{$array$}
    \State $n \gets \text{length of } array$
    \State $pivot \gets \text{random element of } array$
    \If {$n < 2$} \Return
    \EndIf
    \State $l \gets 0, r \gets n$
    \For {$i \gets 0 \text{ to } n-1$}
      \If {$array[i] < pivot$}
        \State $\text{swap } array[i] \text{ and } array[l]$
        \State $l \gets l + 1$
      \EndIf
    \EndFor
    \For {$i \gets n-1 \text{ to } 0$}
      \If {$array[i] > pivot$}
        \State $\text{swap } array[i] \text{ and } array[r - 1]$
        \State $r \gets r - 1$
      \EndIf
    \EndFor
    \State \Call{QuickSort}{$array[0..l]$}
    \State \Call{QuickSort}{$array[r..n]$}
  \EndProcedure
  \end{algorithmic}
\end{algorithm}
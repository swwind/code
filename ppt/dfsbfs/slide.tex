\documentclass[12pt,aspectratio=169]{beamer}

\usepackage{minted}

\usetheme[progressbar=frametitle, numbering=fraction]{metropolis}
\usepackage{appendixnumberbeamer}

\usepackage{booktabs}
\usepackage[scale=2]{ccicons}

\usepackage{pgfplots}
\usepgfplotslibrary{dateplot}

\usepackage{xspace}
\newcommand{\themename}{\textbf{\textsc{metropolis}}\xspace}

% Chinese Fonts (Fontset: fandol,ubuntu)
\usepackage[fontset=fandol]{ctex}

% Math Fonts
\usefonttheme{professionalfonts} 
\usepackage{mathspec}
% \setsansfont[BoldFont={Fira Sans},Numbers={OldStyle}]{Fira Sans Light}
% \setmathsfont(Digits)[Numbers={Lining, Proportional}]{Fira Sans Light}

% Change Color of the theme
\usepackage{xcolor}
\definecolor{DarkGrey}{HTML}{353535}
\definecolor{ECNURed}{RGB}{164,31,53}
\definecolor{ECNUBrown}{RGB}{134,117,77}
\setbeamercolor{normal text}{ fg= DarkGrey  }
\setbeamercolor{alerted text}{ fg= ECNURed  }
\setbeamercolor{example text}{ fg= ECNUBrown  }

% Bolder Fonts for presenting in a large room 
\setsansfont[BoldFont={Fira Sans SemiBold}]{Fira Sans Book}

\title{搜索算法}
\subtitle{深度优先与广度优先搜索}
\date{\today}
\author{罗江楠}
\institute{哈尔滨工业大学(威海)}
\titlegraphic{\hfill\includegraphics[height=1cm]{hitwh.png}}

\begin{document}

\maketitle

\begin{frame}{目录}
  \setbeamertemplate{section in toc}[sections numbered]
  \tableofcontents[hideallsubsections]
\end{frame}

\section[引言]{Introduction}

\begin{frame}[fragile]{引言}
  其实我也不知道要说什么\pause

  就祝大家新年快乐吧
\end{frame}

\section[深度优先搜索]{Depth First Search}

\begin{frame}[fragile]{深度优先搜索}
  \begin{itemize}
    \item 尽可能深地搜索树的分支
    \item 直到所有可能情况都被完全搜索一遍
  \end{itemize}
\end{frame}

\begin{frame}[fragile]{例题:整数分解}
  \begin{itemize}
    \item 求解有多少种将正整数 $n$ 分解为 $3$ 个正整数相加的方案
    \item 两种分解方案不同当且仅当加数中的其中一个数字不相同
    \item 例如 $4=1+1+2=1+2+1=2+1+1$ 共有 $3$ 种不同的分解方式
  \end{itemize}
\end{frame}

\begin{frame}[fragile]{例题:整数分解}
  考虑你们之前学过的枚举算法
  \begin{minted}{cpp}
for (int i = 1; i <= n; ++ i)
  for (int j = 1; j <= n; ++ j)
    for (int k = 1; k <= n; ++ k)
      if (i + j + k == n)
        printf("%d = %d + %d + %d\n", n, i, j, k);
  \end{minted}
  \pause
  时间复杂度 $O(n^3)$
\end{frame}

\begin{frame}[fragile]{例题:整数分解}
  可以考虑优化枚举的上下界,可以将复杂度降低到 $O(n^2)$
  \begin{minted}{cpp}
for (int i = 1; i <= n - 2; ++ i)
  for (int j = 1; j <= n - i - 1; ++ j)
    printf("%d = %d + %d + %d\n", n, i, j, n - i - j);
  \end{minted}
\end{frame}

\begin{frame}[fragile]{例题:整数分解}
  \begin{itemize}
    \item 但是如果将题目改成分解成 $4$ 个正整数的和呢?\pause
    \item 写 $4$ 重 \verb|for| 循环?\pause
    \item 如果改成 $m$ 个正整数呢?\pause
    \item 写 $m$ 重 \verb|for| 循环?\pause
    \item 此时就需要运用这个被称为\textbf{深度优先搜索}的算法了
  \end{itemize}
\end{frame}

\begin{frame}[fragile]{深度优先搜索算法}
  深度优先搜索算法一般使用递归的算法,遵从以下流程
  \begin{itemize}
    \item 首先判断搜索的当前状态是否满足\textbf{结束条件}
    \item 如果结束满足则输出当前状态并返回
    \item 接下来考虑枚举当前状态的每一个可能的下一个状态的转移
    \item 将该转移应用到当前的状态中
    \item 向下递归搜索
    \item 将当前状态恢复为转移前的状态
  \end{itemize}
\end{frame}

\begin{frame}[fragile]{例题:整数分解}
  \begin{minted}{cpp}
int ans[100]; // 用于储存当前分解方案的数组
void dfs(int x, int k) { // 将 x 分解成 k 个正整数
  if (k == 1) { // 搜索的结束条件
    ans[k] = x; output(ans); // 输出方案
    return;
  }
  for (int i = 1; i <= x - k + 1; ++ i) {
    ans[k] = i; // 记录方案
    dfs(x - i, k - 1); // 递归搜索
  }
}
  \end{minted}
\end{frame}

\begin{frame}[fragile]{例题:整数分解}
  \begin{figure}
    \includegraphics[height=200pt]{example1.png}
  \end{figure}
\end{frame}

\begin{frame}[fragile]{复杂度分析}
  每个答案都只会被枚举到一次,因此算法的时间复杂度与答案规模相同\pause

  $$O({n-1 \choose m-1})$$
\end{frame}

\begin{frame}[fragile]{例题:全排列}
  求 $n$ 个元素的全排列。

  如 $n=3$ 时

  \begin{verbatim}
  1 2 3
  1 3 2
  2 1 3
  2 3 1
  3 1 2
  3 2 1
  \end{verbatim}
\end{frame}


\begin{frame}[fragile]{例题:全排列}
  \begin{minted}{cpp}
int vis[100]; // vis[i] 表示数字 i 是否已经用过
int ans[100]; // 用于保存当前的排列状态
void dfs(int depth) { // 参数 depth 表示当前递归的深度
  if (depth == n + 1) { // 如果所有数字都选完了
    output(ans); return; // 输出并返回
  }
  for (int i = 1; i <= n; ++ i) { // 枚举下一个数字
    if (vis[i]) continue; // 如果数字已经被使用过则跳过
    vis[i] = 1; ans[depth] = i; // 使用数字 i
    dfs(depth + 1);
    vis[i] = 0; // ans[depth] = 0; // 恢复原来的状态
  }
}
  \end{minted}
\end{frame}

\begin{frame}[fragile]{复杂度分析}
  总共会递归 $n$ 层,每层都会枚举 $n$ 个数字\pause

  $$O(n^n)$$\pause

  能不能优化到 $O(n!)$ 呢
\end{frame}

\begin{frame}[fragile]{例题:全排列(优化)}
  \begin{minted}{cpp}
std::queue<int> qu; // 储存剩余的数字
std::vector<int> st; // 保存当前的排列的栈
void dfs() { // 甚至不需要深度参数
  // 如果待选数字为空则终止搜索
  if (qu.empty()) return output(st);
  for (size_t i = 0; i < qu.size(); ++ i) {
    // 将队首元素加入到答案栈中并从队列中弹出
    st.push_back(qu.front()); qu.pop();
    dfs(); // 递归搜索下一个元素
    // 将栈顶元素加入到队尾并从栈中弹出
    qu.push(st.back()); st.pop_back();
  }
}
  \end{minted}
\end{frame}

\begin{frame}[fragile]{例题:八皇后问题}
  如何能够在 $8 \times 8$ 的国际象棋棋盘上放置八个皇后,使得任何一个皇后都无法直接吃掉其他的皇后?

  \begin{figure}
    \includegraphics[height=100pt]{killer_queen.jpeg}
    \includegraphics[height=100pt]{queen.png}
  \end{figure}
\end{frame}

\begin{frame}[fragile]{例题:八皇后问题}
  \begin{itemize}
    \item 枚举每个格子里是否有皇后,然后判断是否有冲突。\pause
    \item 复杂度 $O(2^{64})$ 原地爆炸\pause
    \item 枚举每行皇后的位置,然后判断是否有冲突。\pause
    \item 复杂度 $O(8^8)$,依旧有优化的空间
  \end{itemize}
\end{frame}

\begin{frame}[fragile]{例题:八皇后问题}
  \begin{itemize}
    \item 考虑在枚举每一个皇后的位置的时候同时检测是否有冲突,避免无用的搜索。\pause
    \item 具体可以利用数组 \verb|a[i][j]| 表示位置为 $(i,j)$ 的格子是否会被已经存在的皇后吃掉。\pause
    \item 每当放置一个皇后的时候,都需要更新可能会被该皇后吃掉的格子的 \verb|a[i][j]| 的值。\pause
    \item 当回溯的时候,可以恢复 \verb|a[i][j]| 的值。
  \end{itemize}
\end{frame}

\begin{frame}[fragile]{例题:八皇后问题}
  \begin{itemize}
    \item 具体为了简便更新和恢复 \verb|a[i][j]| 的值,每当我们放置皇后的时候就将影响到的格子的 \verb|a[i][j]| 的值加一,每当回溯的时候就将 \verb|a[i][j]| 的值减一。\pause
    \item 这样当且仅当 \verb|a[i][j]| $ = 0$ 时,该位置的格子才有可能被放置下一个皇后,否则就不再枚举该格子是皇后的可能性。
  \end{itemize}
\end{frame}

\begin{frame}[fragile]{例题:八皇后问题}
  \begin{minted}{cpp}
int a[8][8], ans = 0;
void effect(int x, int y, int delta) {
  for (int i = 0; i < 8; ++ i)
    for (int j = 0; j < 8; ++ j)
      if (i == x || j == y || abs(i - x) == abs(j - y))
        a[i][j] += delta;
}
void apply_queen(int x, int y) { effect(x, y, 1); }
void recover_queen(int x, int y) { effect(x, y, -1); }
  \end{minted}
\end{frame}

\begin{frame}[fragile]{例题:八皇后问题}
  \begin{minted}{cpp}
void dfs(int n) {
  // 如果已经放置了八个皇后,则输出结果
  if (n == 8) { ++ ans; return; }
  for (int i = 0; i < 8; ++ i) {
    for (int j = 0; j < 8; ++ j) {
      if (a[i][j] == 0) {
        apply_queen(i, j); // 放置皇后
        dfs(n + 1);
        recover_queen(i, j); // 回溯
      }
    }
  }
}
  \end{minted}
\end{frame}

\begin{frame}[fragile]{例题:八皇后问题}
  \begin{minted}{cpp}
int main() {
  dfs(0);
  std::cout << ans << std::endl;
  // 输出结果:3709440
}
  \end{minted}
  \pause
  可是答案不是 $92$ 么\pause

  显然我们枚举的时候没有考虑每个皇后的先后顺序,导致 $92$ 种不同的解都会被统计 $8!$ 次,因此答案为 $92 \times 8! = 3709440$
\end{frame}

\begin{frame}[fragile]{例题:八皇后问题}
  为了避免这种问题,我们可以在 dfs 的时候顺便带入一个 $(x, y)$ 坐标表示当前最后放置的皇后的位置,之后的枚举仅从这个位置的下一个格子开始。\pause

  稍微简化一点,第 $n$ 个皇后的位置必定在第 $n$ 行,因此我们在 DFS 中只需要枚举一行即可。
\end{frame}

\begin{frame}[fragile]{例题:八皇后问题(优化)}
  \begin{minted}{cpp}
void dfs(int n) {
  if (n == 8) { ++ ans; return; }
  for (int i = 0; i < 8; ++ i) {
    if (a[n][i] == 0) {
      apply_queen(n, i);
      dfs(n + 1);
      recover_queen(n, i);
    }
  }
} // dfs(0); std::cout << ans << std::endl; // 92
  \end{minted}
  \pause
  还能再快一点么
\end{frame}

\begin{frame}[fragile]{例题:八皇后问题(优化)}
  \begin{itemize}
    \item 考虑优化 \verb|apply_queen| 和 \verb|recover_queen| 的过程。
    \item 观察到总共只有 $64$ 个格子,每个格子我们只关心他的 0/1 状态。
    \item 因此我们可以把每个状态压缩到一个 \verb|unsigned long long| 里面。
    \item 预处理每个格子放置皇后会产生冲突的格子,转移的时候只需要将当前状态或上该转移状态即可。
  \end{itemize}
\end{frame}

\begin{frame}[fragile]{例题:八皇后问题}
  \begin{minted}{cpp}
uint64_t t[64]; // 转移状态
void init() {
  for (int x = 0; x < 8; ++ x)
    for (int y = 0; y < 8; ++ y)
      for (int i = 0; i < 8; ++ i)
        for (int j = 0; j < 8; ++ j)
          if (i == x || j == y || abs(i - x) == abs(j - y))
            t[(x << 3) | y] |= 1ull << ((i << 3) | j);
}
// O(n^4),当然也可以直接打表
  \end{minted}
\end{frame}

\begin{frame}[fragile]{例题:八皇后问题}
  \begin{minted}{cpp}
int ans = 0;
void dfs(int n, uint64_t state) {
  if (n == 8) { ++ ans; return; }
  for (int i = 0; i < 8; ++ i) {
    if (state & (1ull << ((n << 3) | i))) continue;
    dfs(n + 1, state | t[(n << 3) | i]);
  }
}
// dfs(0, 0); printf("%d\n", ans); // 92
  \end{minted}
\end{frame}

\subsection[剪枝优化]{Decision Tree Pruning}

\begin{frame}[fragile]{剪枝优化}
  剪枝(Decision Tree Pruning, 决策树剪枝)是指在保证\textbf{正确性}的前提下,采取的一些手段使算法跳过一定不含有目标状态/最优解的分支,从而保证算法更高效迅速地找出目标解的状态。\pause

  一般的入手点有:

  \begin{itemize}
    \item \textbf{优化搜索顺序}:优先从比较优的转移状态开始搜索
    \item \textbf{可行性剪枝}:当前状态不含有目标状态,则直接跳过该状态的所有转移
    \item \textbf{最优性剪枝}:当前状态不含有更优解,则直接跳过该状态的所有转移
    \item \textbf{记忆化搜索}:对于一些纯函数的搜索,可以将搜索过的状态记录下来,避免重复搜索
  \end{itemize}
\end{frame}

\begin{frame}[fragile]{例题:[BeiJing2011]符环}
  给出一个长度为 $n(1 \le n \le 50)$ 的字符串 $S$,字符串仅包含 \verb|'S'| 和 \verb|'D'| 两种字母。

  询问有多少组合法的长度为 $2n$ 的括号序列 $R$ 满足:$(0 \le i < n)$

  \begin{itemize}
    \item 若 $S_i$ 为 \verb|'S'|,则 $R_i = R_{i+n}$
    \item 若 $S_i$ 为 \verb|'D'|,则 $R_i \ne R_{i+n}$
  \end{itemize}

  例如 $S$ 为 \verb|"SSD"|,则满足条件的 $R$ 有且仅有一种可能的解为 \verb|"()(())"|。\pause

  提示:枚举每一个 $R_i$,然后检查 $R$ 是否满足条件,复杂度为 $O(2^n \cdot n)$,考虑如何优化。
\end{frame}

\begin{frame}[fragile]{例题:[BeiJing2011]符环}
  DFS 的时候记录 $4$ 个状态,分别为:

  \begin{itemize}
    \item 当前搜索到的深度 $d$ 的位置
    \item 当前 $R[0:n]$ 中剩余未匹配的左括号数量
    \item 当前 $R[n:2n]$ 中左侧未匹配的右括号数量
    \item 当前 $R[n:2n]$ 中右侧未匹配的左括号数量
  \end{itemize}

  \begin{verbatim}
0      n-1 | n         2n-1
..(((..... | )))..(((......
  ^^^--- x | ^^^--- y
           |      ^^^--- z
  \end{verbatim}
\end{frame}

\begin{frame}[fragile]{例题:[BeiJing2011]符环}
  终止条件即为 $d = n$,同时我们返回当前的状态是否是一个合法的状态。

  \begin{minted}{cpp}
int dfs(int d, int x, int y, int z) {
  if (d == n) {
    return x == y && !z;
  }
  int ans = 0;
  \end{minted}

  接下来我们分情况讨论 $S_i$ 的类型。
\end{frame}

\begin{frame}[fragile]{例题:[BeiJing2011]符环}
  \begin{minted}{cpp}
  if (S[d] == 'S') {
    // put '('
    ans += dfs(d + 1, x + 1, y, z + 1);
    // put ')'
    if (x) {
      if (z) {
        ans += dfs(d + 1, x - 1, y, z - 1);
      } else {
        ans += dfs(d + 1, x - 1, y + 1, z);
      }
    }
  }
  \end{minted}
\end{frame}

\begin{frame}[fragile]{例题:[BeiJing2011]符环}
  \begin{minted}{cpp}
  if (S[d] == 'D') {
    // put '('
    if (z) {
      ans += dfs(d + 1, x + 1, y, z - 1);
    } else {
      ans += dfs(d + 1, x + 1, y + 1, z);
    }
    // put ')'
    if (x) {
      ans += dfs(d + 1, x - 1, y, z + 1);
    }
  }
  \end{minted}
\end{frame}

\begin{frame}[fragile]{复杂度分析}
  \begin{itemize}
    \item 首先复杂度十分玄学,不好直接利用公式量化计算。
    \item 但是我们可以通过观察,发现该算法相当于枚举了答案序列中的每一位。
    \item 因此该算法的复杂度是 $O(2^n)$,还能如何优化呢?
  \end{itemize}
\end{frame}

\begin{frame}[fragile]{复杂度分析}
  \begin{itemize}
    \item 观察发现 \verb|dfs(d, x, y, z)| 的返回值只与四元组 $(d, x, y, z)$ 有关,满足纯函数的定义。
    \item 而四元组 $(d, x, y, z)$ 可能的取值只有 $n^4$ 的情况。
    \item 因此利用记忆化搜索,该算法的复杂度可以降低到 $O(n^4)$。
  \end{itemize}
\end{frame}

\begin{frame}[fragile]{例题:[BeiJing2011]符环}
  \begin{minted}{cpp}
int f[52][52][52][52];
void init() { // 初始化记忆化缓存
  memset(f, -1, sizeof(f));
}
int dfs(int d, int x, int y, int z) {
  if (d == n) return x == y && !z;
  int &ans = f[d][x][y][z];
  if (~ ans) return ans;
  ans = 0;
  // 下略
}
  \end{minted}
\end{frame}

\section[广度优先搜索]{Breadth First Search}

\begin{frame}[fragile]{广度优先搜索}
  \begin{itemize}
    \item 就是每次都尝试访问同一层的节点
    \item 如果同一层都访问完了,再访问下一层
  \end{itemize}
\end{frame}

\begin{frame}[fragile]{例题:走迷宫}
  给出一张 $n \times m$ 的地图,其中 \verb|'#'| 表示墙,\verb|'.'| 表示可以走的路,\verb|'S'| 表示起点,\verb|'E'| 表示终点。

  请求出走迷宫的最短路径。

  \begin{verbatim}
##########
#S.......#
########.#
#T.......#
##########
  \end{verbatim}
\end{frame}

\begin{frame}[fragile]{例题:走迷宫}
  \begin{minted}{cpp}
int maze[52][52], dx = {0, 0, 1, -1}, dy = {1, -1, 0, 0};
int dfs(int x, int y) {
  if (x == tx && y == ty) return 0;
  int ans = INT_MAX;
  for (int i = 0; i < 4; ++i) {
    int nx = x + dx[i], ny = y + dy[i];
    if (inside_maze(nx, ny) && maze[nx][ny] != '#') {
      ans = min(ans, dfs(nx, ny) + 1);
    }
  }
  return ans;
}
  \end{minted}
\end{frame}

\begin{frame}[fragile]{例题:走迷宫}
  前面的 DFS 算法虽然可以解决一小部分的数据,但是如果遇到一些极端情况,则会导致复杂度暴增,甚至发生无限递归循环。

  这时候就应该需要使用广度优先搜索,每次一层一层的访问,直到找到终点。
\end{frame}


\begin{frame}[fragile]{广度优先搜索}
  广度优先搜索算法一般遵从以下几个步骤:
  
  \begin{itemize}
    \item 初始化一个队列,将起点或初始状态放入队列中
    \item 每次从队列中取出队首的状态,并对其进行处理
    \item 如果该状态是目标状态,则结束搜索
    \item 否则将该状态的所有后继未访问状态放入队列中
    \item 如果直到队列未空也没有找到目标状态,则说明无解
  \end{itemize}
\end{frame}

\begin{frame}[fragile]{例题:走迷宫}
  \begin{minted}{cpp}
char maze[52][52]; // 储存迷宫地图
int dis[52][52]; // 记录每个点到起点的距离
int dx = {0, 0, 1,-1}; // 四个方向的 delta x
int dy = {1,-1, 0, 0}; // 四个方向的 delta y
std::queue<std::pair<int, int>> q;
void bfs() {
  // 初始化起点
  q.push(std::make_pair(sx, sy));
  memset(dis, -1, sizeof(dis));
  dis[sx][sy] = 0;
  // 见下页
}
  \end{minted}
\end{frame}

\begin{frame}[fragile]{例题:走迷宫}
  \begin{minted}[fontsize=\small]{cpp}
while (!q.empty()) {
  int x = q.front().first, y = q.front().second;
  q.pop(); // 取出队首元素
  if (x == tx && y == ty) return dis[x][y];
  for (int i = 0; i < 4; ++ i) { // 枚举每一个方向
    int nx = x + dx[i], ny = y + dy[i];
    if (inside_maze(nx, ny) && maze[nx][ny] != '#'
        && dis[nx][ny] == -1) {
      dis[nx][ny] = dis[x][y] + 1;
      q.push(make_pair(nx, ny));
    }
  }
}
  \end{minted}
\end{frame}

\section[例题]{Examples}

\end{document}